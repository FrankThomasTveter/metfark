% Changing book to article will make the footers match on each page,
% rather than alternate every other.
%
% Note that the article class does not have chapters.
\documentclass[letterpaper,10pt,twoside,twocolumn,openany]{book}

% Use babel or polyglossia to automatically redefine macros for terms
% Armor Class, Level, etc...
% Default output is in English; captions are located in lib/dndstring-captions.sty.
% If no captions exist for a language, English will be used.
%1. To load a language with babel:
%	\usepackage[<lang>]{babel}
%2. To load a language with polyglossia:
%	\usepackage{polyglossia}
%	\setdefaultlanguage{<lang>}
\usepackage[english]{babel}
%usepackage[italian]{babel}
% For further options (multilanguage documents, hypenations, language environments...)
% please refer to babel/polyglossia's documentation.

\usepackage[utf8]{inputenc}
\usepackage{lipsum}
\usepackage{listings}
\usepackage{hyperref}

% dnd package options 
% bg-full   : Default option. Use paper background and fancy footer.
% bg-print  : Use fancy footer but not background.
% bg-none   : No paper background and plain footer.
% justified : Use full justification for text layout instead of ragged right.
\usepackage{dnd}

\lstset{%
  basicstyle=\ttfamily,
  language=[LaTeX]{TeX},
}

% Start document
\begin{document}

% Your content goes here

% Comment this out if you're using the article class.
\chapter{File ARchive Kit (FARK)\tiny{met.no, 2018}}

\section{Introduction}
The process of generating {\bf verification results} by co-locating {\bf observation} and {\bf model data}, 
typically requires 200 pieces of information.
The FARK system provides a web-interface \url{http://fark.met.no} for the user to specify this information and
organize the regular FARK production of verification result.

\begin{quotebox}
   Use FARK if you need regular production of {\bf verification results} 
   and don't want to spend your time writing scripts.
\end{quotebox}

\subsection{FARK production}

FARK first generates {\bf indexed lists} of the {\bf NetCDF} model files 
and {\bf BUFR} observation files.
\begin{quotebox}
File indexes are usually sorted by the epoch-time.
\end{quotebox}
A basic description of the file formats is available in the \hyperlink{appendix}{{\em appendix}}.

Next, model fields are interpolated to relevant observation locations and time.
This co-located data is written to a {\bf table file} according to the specifications in a {\bf plotting script}.

Finally, the {\bf plotting script} produces the {\bf verification plots}.

\begin{quotebox}
Verification results are found under:
\href{/lustre/storeA/project/nwp/fark/.}{\lstinline!/lustre/storeA/project/nwp/fark!}
\end{quotebox}

\section{Web interface}
All information necessary to generate verification results can
be put into the FARK web interface.
The web interface contains buttons designed to make it easier for the
user to provide this information.

\subsection{Buttons}

\begin{dndtable}[cX][DmgCoral]
  \textbf{Button} & \textbf{Description} \\
  \raisebox{-0.2\height}{\includegraphics[height=13pt]{server.jpg}}& show alternatives\\
  \raisebox{-0.2\height}{\includegraphics[height=13pt]{left.jpg}}  & move information\\
  \raisebox{-0.2\height}{\includegraphics[height=13pt]{minus.jpg}} & delete table entry\\
  \raisebox{-0.2\height}{\includegraphics[height=13pt]{plus.jpg}}  & add table entry\\
  \raisebox{-0.2\height}{\includegraphics[height=13pt]{test.jpg}}  & test process\\
  \raisebox{-0.2\height}{\includegraphics[height=13pt]{run.jpg}}   & start process\\
  \raisebox{-0.2\height}{\includegraphics[height=13pt]{stop.jpg}}  & stop process\\
  \raisebox{-0.2\height}{\includegraphics[height=13pt]{save.jpg}}  & save setup to server
\end{dndtable}

\begin{quotebox}
Press the blue ``tab title'' to reload data from server.
\end{quotebox}

The web interface is further divided into the following tabs: \lstinline!Model!, \lstinline!Observation!, \lstinline!Colocation!, 
\lstinline!Plot! and \lstinline!Auto!.

\subsection{Model tab}

\begin{paperbox}{Model tab}
  \includegraphics[width=\columnwidth]{fark_model.jpg}
\end{paperbox}
In this tab you specify which \hyperlink{netcdf}{NetCDF} model files to index and which variable to use when sorting the  {\bf model index}.
The index sorting variable is given a {\bf model index target} name.

\begin{quotebox}
  A {\bf target} name is a 'short name' used to
  represent the model variable or observation 
  parameter.
\end{quotebox}

\subsection{Observation tab}
\begin{paperbox}{Observation tab}
  \includegraphics[width=\columnwidth]{fark_obs.jpg}
\end{paperbox}
Here you specify which \hyperlink{bufr}{BUFR} observation files to index and the {\bf expression} used to sort the {\bf observation index}.
The index expression is assigned a {\bf target} name.
The user must also define the {\bf observation targets} that are used in the index expression.
The observation targets point to specific positions in the \hyperlink{sequence}{BUFR sequence}.

\subsection{Colocation tab}
This is where you specified how the model and 
observation data should be matched.
The colocation tab contains several tables.

\subsubsection{Model targets table}
\begin{paperbox}{Model targets table}
  \includegraphics[width=\columnwidth]{coloc_model.jpg}
\end{paperbox}
The {\bf model targets} table lists model variables and their target names.
The (saved) {\bf model index} target name is listed first.
\begin{quotebox}
  Variables in red are not in the scanned model file.
\end{quotebox}
You may also use a {\bf dimension} in your {\bf model target}. 
In this case enter the dimension name surrounded by brackets instead of a variable name,
for instance \lstinline!(ensemble_member)!.
\begin{paperbox}{Model variable offset}
  \includegraphics[width=\columnwidth]{offset.jpg}
\end{paperbox}
If you want a model variable, say 24 hours earlier than the observation time, you can use an {\bf offset}.
The square brackets added after the variable name should contain the 
name of the model target that should be offset and the offset amount (seperated by colon).
In this example model target \lstinline!precip_acc_m24! contains the accumulated precipitation 24 
hours before the target \lstinline!precip_acc!.
\hyperlink{matching}{\em Matching rules}) should not be defined for offset model targets .

\subsubsection{Observation targets table}
\begin{paperbox}{Observation targets table}
  \includegraphics[width=\columnwidth]{coloc_obs.jpg}
\end{paperbox}
In the {\bf observation targets} table the user can specify observation targets in the BUFR sequence.
The targets already defined in the (saved) {\bf observation index} are listed first. 
Additional observation targets needed to match the observation with the model fields are added here.

\subsubsection{Position variables}
In the example above \lstinline!obs_wdir! uses a {\bf position variable}, \lstinline!W! 
instead of a number. 
The reason for this is that the wind speed happens to appear after a \hyperlink{delayed}{\em delayed replicator} 
in the BUFR sequence.
The FARK system will search the BUFR sequence for the specified descriptor, \lstinline!11001!,
and assign the corresponding position to the {\bf position variable},  \lstinline!W!. 
The {\bf position variable} can be used in the position expressions of later observation targets,
as we see in the example with \lstinline!obs_wspd! with the position expression \lstinline!W+1!.

\begin{quotebox}
   Use {\bf position variables} when you process radiosonde TEMP BUFR messages.
\end{quotebox}

If only the descriptor is specified, the system will search the BUFR sequence 
for the next entry with the given descriptor.

\subsubsection{Internal variables}
Internal variables are indicated as position variables in the position field,
without any descriptor.
\begin{dndtable}[cX][DmgCoral]
  \textbf{Position} & \textbf{Description} \\
  mid &  model file index position\\
  oid &  observation file index position \\
  bid &  BUFR message number \\
  sid &  observation number in message \\
  lid &  location number in message
\end{dndtable}

\begin{quotebox}
A location is identified using {\bf oid}, {\bf bid} and {\bf lid}.
\end{quotebox}

\subsubsection{Duplicate location}
If you want to compare the same observation location to several model fields, for instance different ensemble members,
you need to duplicate the observation location.
In this case you specify the min and max values and not the position nor descriptor
(the max value may be a model dimension).

\begin{quotebox}
   Duplicate locations if you need to process multiple {\bf model ensemble members}.
\end{quotebox}
The observation target \lstinline!obs_hybrid! in the example above is an example of location duplication.
The target \lstinline!obs_hybrid! takes the value of the duplication index, i.e. \lstinline!1,2,3,...,hybrid0!.

\hypertarget{matching}{}
\subsubsection{Match rules table}
The {\bf match rules} table specifies how the model targets should match the
observation targets. 
\begin{paperbox}{Match rules table}
  \includegraphics[width=\columnwidth]{coloc_match.jpg}
\end{paperbox}
Model targets with blank observation target expressions are not used for matching.
If insufficient matching rules are specified so that FARK can not determine how to
interpolate a dimension used by a model target, FARK will average over that dimension (if it is small).

\begin{quotebox}
A location is discarded if a match rule has a target that is undefined.
\end{quotebox}

\hypertarget{default}{}
\subsubsection{Default table}
The {\bf default} table is only visible if the observation index file has been set to \lstinline!<none>!.
The default table specifies how the model targets should match default values.
\begin{paperbox}{Default table}
  \includegraphics[width=\columnwidth]{default.jpg}
\end{paperbox}

\subsubsection{filters}

Co-location takes a lot of computer resources, and it is therefore a good strategy
to filter out unwanted data as early as possible in the data processing.

The observation {\bf min} and {\bf max} filters are applied immediately to discard locations
when the BUFR messages are read from file. 

The {\bf observation filter} expression is applied when all locations in a BUFR message have been read from file,
and this filter allows functions that apply to the whole message. 
\begin{quotebox}
  The {\bf observation filter} expression can only be based on observation targets.
\end{quotebox}
For instance \lstinline!msgclosest!
selects the location that produces an expression closest to a list
\begin{lstlisting}
   msgclosest(obs_pres*0.01,1000,500)
\end{lstlisting}

The model {\bf min} and {\bf max} filters are applied immediately when the location has been interpolated.

The {\bf model filter} expression is applied when all relevant model fields have been interpolated to the observation location.
This is the most expensive filter in terms of computer resources.

There is a debug option available for testing the built in filter functions.
The debug expression can not accept any targets.
Note that ``blank'' returns zero.

\begin{quotebox}
  A location is rejected if a filter expression returns 0.
\end{quotebox}


\subsection{Plot tab}

\begin{paperbox}{Plot tab}
  \includegraphics[width=\columnwidth]{fark_plot.jpg}
\end{paperbox}

This is where you provide information requested by the plotting script.
The user chooses plotting script in the \lstinline!Category! field.

Verification results are placed on disk according to the \lstinline!Output table file! and \lstinline!Output prefix! paths.
Use \lstinline!YY MM DD HH MI! as wildcards in the output paths.

There are two tables in the plot tab. 
The {\bf Attributes} table allows the user to specify attributes that apply for all the data,
for instance titles, units and labels.
Some attributes can only have fixed values, indicated by an action button
 \raisebox{-0.2\height}{\includegraphics[height=13pt]{server.jpg}},
and some attributes are used to add attributes and columns.

A plotting script can compare different colocation datasets.
The {\bf Dataset} table assigns every dataset an \lstinline!Id!, \lstinline!Colocation file! and 
\lstinline!legend!, along with the column-expressions requested by the plotting script.

\subsection{Auto tab}
\begin{paperbox}{Auto tab}
  \includegraphics[width=\columnwidth]{auto.jpg}
\end{paperbox}

This is where you tell the computer to actually do some work.
Available types of jobs are:
\begin{dndtable}[cX][DmgCoral]
 {\bf Type} & {\bf Description} \\
 model & maintain model index file, \\
 obs & maintain observation index file, \\
 coloc & {\em debugging for advanced users,} \\
 plot & generate verification products.
\end{dndtable}
Each job can be executed manually or according to a schedule.

%The task configuration can be {\bf tested} manually by pressing \includegraphics[height=13pt]{test.jpg}. 
%Tasks can be {\bf executed} manually by pressing \includegraphics[height=13pt]{run.jpg}.
%A running task can be {\bf stopped} manually by pressing \includegraphics[height=13pt]{stop.jpg}.

\section{Performance}

Co-locating large amounts of data takes a lot of resources, and it is quite easy to
set up jobs that are not able to run successfully. Generating {\bf table files}
does not usually require too much memory but can take much processing time.
However the {\bf plotting script} will typically not be able to process a {\bf table file} with 
more than 10 million entries without running out of memory (on Ares). 
This corresponds to 20 days of verification for northern Europe.
Although FARK has all the features necessary for a full verification of
operational forecast ensemble data using radiosonde data, this is option is
probably not realistic from a computer resource perspective.

\newpage
\hypertarget{appendix}{}
\section{Appendix}
\hypertarget{netcdf}{}
\subsection{NetCDF model files}
The NetCDF model files each contain a set of dimensions and variables, 
where each variable may have zero or more dimensions, for instance \lstinline!latitude(x,y)! where
\lstinline!x! and \lstinline!y! are dimensions. 
Note that some variables are accumulated, for instance \lstinline!precipitation_amount_acc(...,time)!. 
Rain rate is calculated by differentiating this variable with respect to time.

\hypertarget{bufr}{}
\subsection{BUFR observation files}
A BUFR observation file may contain many BUFR messages with different BUFR type and sub-type.
Each BUFR message may contain many observations of the same type, for instance SYNOP or TEMP. 
An observation may further contain many locations, for instance a radiosonde TEMP 
observation may contain data from many different heights in the atmosphere.


\hypertarget{sequence}{}
\subsection{BUFR sequence}
BUFR observations with the same BUFR type and sub-type use the same {\bf BUFR sequence}.
\begin{paperbox}{BUFR sequence example}
  \includegraphics[width=\columnwidth]{bufr1.jpg}
\end{paperbox}
The BUFR sequence contains a {\bf position}, {\bf descriptor} and {\bf value}
for each parameter in the observation.
The {\bf descriptor} is used to identify the observation parameter, for instance 
\lstinline!pressure! is identified by the descriptor \lstinline!7004!.
\hypertarget{delayed}{}
\subsection{Delayed replicator}
\begin{paperbox}{Delayed replicator}
  \includegraphics[width=\columnwidth]{bufr2.jpg}
\end{paperbox}

A BUFR sequence may contain a {\bf delayed replicator} (descriptor \lstinline!31001!), 
which will duplicate a sub-section of the BUFR sequence the specified number of times. 

\begin{quotebox}
  BUFR sequence positions after a {\bf delayed replicator} are not ``fixed''.
\end{quotebox}

\newpage
\section{HOW TO...}

\subsection{Create a model index}
Change focus to the ``Model'' tab.
\subsubsection{1. Make sure index does not exist}
Enter the name of your new index in the \lstinline!setup file:! field, for instance \lstinline!test.cfg!,
and press  \raisebox{-0.2\height}{\includegraphics[height=13pt]{server.jpg}} next to the field.
\begin{paperbox}{Model tab}
  \includegraphics[width=\columnwidth]{how_mod.jpg}
\end{paperbox}
If your options include \lstinline!<mkfile>! then the index does not exist.
If the index already exists, the index setup will be loaded automatically.
In this case press \lstinline!<rmfile>! to delete the existing index.
Remember to use the correct password when changing or deleting an existing index.
\subsubsection{2. Find a suitable index to copy}
Enter the name of the index you wish to copy in the \lstinline!setup file:! field.
You may press \raisebox{-0.2\height}{\includegraphics[height=13pt]{server.jpg}} to navigate.
The index setup will be loaded automatically when an existing index is specified.
\subsubsection{3. Create new index setup}
Enter the name of the new and non-existing index you wish to create in the \lstinline!setup file:! field.
Select the password that has to provided when changing or deleting the new index.
Press {\raisebox{-0.2\height}{\includegraphics[height=13pt]{save.jpg}} to create the index setup on the server.
\subsubsection{4. Edit the new index setup}
Edit the index setup fields.
You may press \raisebox{-0.2\height}{\includegraphics[height=13pt]{server.jpg}} next to
\lstinline!Model file filter(regexp)! to select which model file to scan.
Variables in the scanned file appear as
relevant options to other fields (when pressing \raisebox{-0.2\height}{\includegraphics[height=13pt]{server.jpg}}),
for instance  \lstinline!Variable:!.
When you are satisfied, press \raisebox{-0.2\height}{\includegraphics[height=13pt]{save.jpg}} to save your settings.
\subsubsection{5. Create the model index}
Switch to the \lstinline!Auto! tab.
Select the \lstinline!model! type, and press \raisebox{-0.2\height}{\includegraphics[height=13pt]{server.jpg}} next to \lstinline!Setup file!.
\begin{paperbox}{Auto tab}
  \includegraphics[width=\columnwidth]{how_auto.jpg}
\end{paperbox}
Your new model setup \lstinline!test.cfg! should now be available as an option, select it.
Finally press \raisebox{-0.2\height}{\includegraphics[height=13pt]{run.jpg}} to the right of your model setup file to create the model index itself.

If you know the password, you can schedule an automatic maintenance of your model
index and press \raisebox{-0.2\height}{\includegraphics[height=13pt]{plus.jpg}}.

\subsection{Create an observation index}
Creating the observation index setup is similar to creating the model index setup detailed above.
However, editing the setup file is somewhat different and explained in some detail here.

Change focus to the ``Observation'' tab.
After creating your new observation index setup, 
you may press \raisebox{-0.2\height}{\includegraphics[height=13pt]{server.jpg}} next to
\lstinline!Obs file filter(regexp)! to select which observation file to scan.
The \lstinline!BUFR type! and \lstinline!Sub type! options are extracted from the scanned file,
along with their associated BUFR sequences.
Define the targets that you need in your observation index expression by entering values
in the bottom row of the \lstinline!observation targets! table, and by pressing the \raisebox{-0.2\height}{\includegraphics[height=13pt]{plus.jpg}} to the right.
You may remove a row by pressing  \raisebox{-0.2\height}{\includegraphics[height=13pt]{minus.jpg}} to the right of the respective row.
The values of the removed row are put in the bottom row.
When you are satisfied, press \raisebox{-0.2\height}{\includegraphics[height=13pt]{save.jpg}} to save your settings.

\newpage
\subsection{Examples}
\begin{paperbox}{FF10m score}
  \centerline{\includegraphics[width=0.8\columnwidth]{ff10m_score_hl.jpg}}
\end{paperbox}

\begin{paperbox}{FF10m vs MSLP}
  \centerline{\includegraphics[width=0.8\columnwidth]{ff10m_mslp_hl.jpg}}
\end{paperbox}

\begin{paperbox}{FF10m profile}
  \centerline{\includegraphics[width=0.75\columnwidth]{ff10m_prof.jpg}}
\end{paperbox}

\begin{paperbox}{FF10m SDE series}
  \centerline{\includegraphics[width=0.8\columnwidth]{ff10m_series_hl.jpg}}
\end{paperbox}

\begin{paperbox}{FF10m MEAN series}
  \centerline{\includegraphics[width=0.85\columnwidth]{ff10m_time_hl.jpg}}
\end{paperbox}

%\begin{paperbox}{FF10m time series}
%  \centerline{\includegraphics[width=0.85\columnwidth]{ff10m_series_hl.jpg}}
%\end{paperbox}

% End document
\end{document}
